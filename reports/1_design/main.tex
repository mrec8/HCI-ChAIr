\documentclass[a4paper]{article}
\usepackage{graphicx}
\usepackage{geometry}
\usepackage{hyperref}
\usepackage{xcolor}
\usepackage{booktabs}
\usepackage{xspace}
\geometry{a4paper, margin=1in}

\newcommand\chair{\textbf{Ch\textcolor[HTML]{6B8ACD}{AI}r}\xspace}

\title{\chair: Brain-Computer Interface for Wheelchair Mobility}
\author{Mart\'i Recalde, Bruno S\'anchez, Zachary Parent}
\date{\today}

\begin{document}

\begin{titlepage}
    \maketitle
    \vspace{2cm}
    \tableofcontents
    \section{Abstract}
    Quadriplegic individuals face significant challenges in mobility due to
    limited control options for wheelchairs. Existing solutions, such as hand,
    head, and sip-and-puff controls, fail to provide universal accessibility.
    This paper presents \chair, a novel Brain-Computer Interface (BCI)-enabled
    wheelchair control system that leverages EEG signal processing and AI to
    create a customizable and adaptive mobility solution. We discuss current
    state-of-the-art control methods, the technological advancements in BCI,
    and our proposed system's potential to revolutionize wheelchair mobility.    
\end{titlepage}


\section{Problem Statement}

The global wheelchair user base exceeds 65 million, with 5 million individuals
suffering from quadriplegia. Traditional control mechanisms rely on residual
motor function, making them unsuitable for users with severe neuromuscular disorders. 

There is a high coincidence of wheelchair usership and quadriplegia globally.
Wheelchairs are a necessity for many quadriplegic individuals, and the current
control mechanisms are not suitable for many quadriplegic individuals.
According to the Journal of NeuroEngineering and Rehabilitation,
\textit{"Chin controls is also an
available control schema but may not be effective
for someone with a neuromuscular disease
that has progressed beyond their ability to
control chin movement"} \cite{neurorehab}

\subsection{Existing Control Mechanisms}
\begin{itemize}
    \item Hand control (joysticks)
    \item Head control (tilt sensors)
    \item Sip-and-puff systems (breath-based control)
    \item Chin control (limited effectiveness for advanced neuromuscular diseases)
\end{itemize}

These approaches are often cumbersome, require extensive training, and do not account
for degenerative conditions where motor function continues to decline.

\section{State of the Art}

Recent advancements in assistive technology have introduced alternative wheelchair
control methods:

\subsection{Advanced Joystick and Voice Control}
Voice commands and AI-assisted joystick input offer improvements in accessibility
but still depend on the user's residual mobility or speech capabilities.

\subsection{Brain-Computer Interfaces (BCI)}
BCI technology interprets neural activity to control external devices,
offering a promising avenue for hands-free and muscle-independent wheelchair navigation.

\subsection{EEG-Based Signal Processing}
BCI systems rely on Electroencephalography (EEG) to capture brain signals.
The processing pipeline consists of:
\begin{itemize}
    \item Signal Acquisition: Raw EEG data collection.
    \item Preprocessing: Noise removal and normalization.
    \item Feature Extraction: Techniques such as Common Spatial Patterns (CSP) and Fast Fourier Transform (FFT).
    \item Classification: Machine learning models (SVM, LDA, CNN) to map EEG patterns to wheelchair commands.
\end{itemize}

\subsection{Challenges in BCI Implementation}
\begin{itemize}
    \item High variability in EEG signals between users.
    \item Noise and artifacts from muscle movements.
    \item Lengthy calibration times.
\end{itemize}

\section{Our Approach}

\chair integrates cutting-edge BCI technologies with machine learning-based
classifiers for real-time wheelchair control, extending accessibility to
a wider range of users, and providing an innovative alternative to conventional control systems.

In this section, we discuss the key components of the \chair system, including:
\begin{itemize}
    \item \textbf{Hardware}: The multiple options for EEG-based BCI devices, and their main features.
    \item \textbf{Usability}: The AI training process and its customization features.
\end{itemize}


\subsection{Hardware}

A key component of the \chair system is the BCI headset employed for the measurement of brain
signals. There is a wide array of EEG-based devices available, each with unique features
and target audiences. The choice of hardware is crucial to the system's performance and
usability; therefore, we must compare a selection of such devices, balancing factors
such as customizability, usability, and data quality:

\begin{itemize}
    \item \textbf{OpenBCI:} OpenBCI is the most \textit{open-source} and \textit{customizable} EEG platform, making it the top choice for \textit{researchers} and \textit{developers} who want full control over both hardware and software. Their flagship models, the Cyton and Cyton + Daisy, offer up to 16 channels of raw EEG data without any subscription or data restrictions. Combined with the Ultracortex Mark IV headset, OpenBCI provides high-precision data acquisition with 3D-printable, modular components. The large community and Python, MATLAB, and OpenBCI GUI integrations make it the most flexible tool for AI model training.
    \item \textbf{Muse:} Muse offers the most \textit{user-friendly} and \textit{accessible} EEG headset, designed for meditation, mindfulness, and \textit{light EEG experimentation}. The Muse 2 and Muse S are both affordable, lightweight, and comfortable, with built-in sensors for EEG, heart rate, and motion tracking. While the device only provides 4 channels of EEG data and requires a paid subscription for raw data access, its ease of use and accompanying app make it one of the best options for non-technical users or beginners looking to explore neurofeedback.
    \item \textbf{Unicorn Hybrid Black:} The Unicorn Hybrid Black stands out for offering \textit{research-grade EEG quality} in a compact, wireless headset. With 8 channels and precise dry electrodes, it is ideal for applications like brain-computer interfaces and machine learning experiments. While more expensive than Muse or NeuroSky, it delivers higher signal quality without requiring extensive setup, making it an excellent option for \textit{researchers who need professional results} without the hassle of building their own hardware.
    \item \textbf{NeuroSky MindWave:} NeuroSky MindWave is the \textit{most affordable} EEG device on the market, making it a great entry-level option for \textit{educational purposes} and \textit{simple brain-computer interface projects}. Its single-channel sensor makes it less precise than other devices, but it remains a popular option for introductory applications like attention monitoring or meditation apps. The data access is limited, but it offers a low-cost gateway into the world of EEG technology.
    \item \textbf{Neurosity:} The Neurosity Crown is the best \textit{developer-focused} EEG headset, designed specifically to integrate with AI applications and cloud-based software. It provides 8 EEG channels and runs on a Linux-based system, allowing developers to build full-stack BCI applications directly from the device. Though the quality of raw data access is somewhat restricted compared to OpenBCI, the seamless API integration and support for JavaScript and cloud computing make it an attractive option for \textit{software engineers} who want to create brain-powered apps quickly.
    \item \textbf{Emotiv:} Emotiv offers the best \textit{balance between research and consumer usability}, with a range of wireless EEG devices for both personal and scientific use. The Emotiv Insight 2 is a lightweight, 5-channel headset geared toward personal wellness, while the EPOC X offers 14 channels of high-quality data for research applications. Emotiv's platform provides an intuitive interface for beginners while still granting access to raw EEG data for advanced users, making it one of the most \textit{versatile} brands on the market.
\end{itemize}

\subsection{Usability}

The \chair system allows users to personalize their wheelchair control experience through
AI-based training. The system supports the following features:

\begin{itemize}
    \item \textbf{Personalized AI Training:} The AI classifier is trained in under one hour, adapting to the user's unique brain patterns.
    \item \textbf{Fast and Easy Setup:} The interface is intuitive and does not require technical expertise.
    \item \textbf{Seamless Integration:} The system automatically manages the connection between the BCI hardware and the wheelchair.
\end{itemize}

This user-centric design prioritizes accessibility, ensuring that the system focuses on
usability rather than technical complexity.

\section{Organization}

The development and deployment of \chair follow a structured timeline:
\begin{table}[ht]
    \centering
    \begin{tabular}{ll}
    \toprule
    Timeline & Milestone \\
    \midrule
    February & Hardware procurement, user studies \\
    March & Virtual testing environment setup \\
    April & AI training backend and interface implementation \\
    May & Beta testing phase \\
    Summer 2025 & Global launch \\
    \bottomrule
    \end{tabular}
    \caption{Project Development Timeline}
\end{table}

Our interdisciplinary team combines expertise in computer science,
aerospace engineering, and mathematics to bring \chair to life.

\section{Conclusion}

\chair represents a significant leap forward in mobility solutions
for quadriplegic individuals. By harnessing BCI technology,
we enable seamless and intuitive wheelchair control, reducing dependency
on physical movement. Future work includes optimizing classification models,
reducing calibration time, and expanding accessibility to
lower-cost EEG solutions.

\begin{thebibliography}{9}
\bibitem{who} World Health Organization, \textit{Guidelines on the Provision of Manual Wheelchairs in Less-Resourced Settings}. \url{https://www.who.int/publications/i/item/guidelines-on-the-provision-of-manual-wheelchairs-in-less-resourced-settings}.
\bibitem{neurorehab} Journal of NeuroEngineering and Rehabilitation, \textit{Advanced Control Strategies for Quadriplegic Wheelchair Users}. \url{https://www.nature.com/articles/sc2012158}.
\bibitem{frontiers} Frontiers in Robotics and AI, \textit{Brain-Computer Interfaces for Assistive Technologies}. \url{https://www.frontiersin.org/journals/robotics-and-ai/articles/10.3389/frobt.2022.885610/full}.
\end{thebibliography}

\end{document}

